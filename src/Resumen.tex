~Durante la última década, los miembros del grupo de investigación integrados en esta propuesta han liderado el desarrollo de nuevas tecnologías para la detección de radiación ionizante basadas en cámaras de gas xenón. Concretamente, el investigador principal (IP) propuso en 2007 el desarrollo de un nuevo tipo de detector llamado NEXT, una cámara de proyección temporal utilizando xenón a alta presión con amplificación electroluminescente de la señal. El principio básico de operación de este detector explota el rápido centelleo con que el xenón responde al paso de la radiación, así como sus propiedades de gas noble para medir la energía y trayectoria de partículas cargadas. Durante los años 2008 a 2014 el IP  ha liderado una colaboración internacional (colaboración NEXT) cuyo objetivo es la construcción de un detector con una masa de 100 kg de xenón enriquecido al 90\% en el isótopo \XE. El objetivo de este detector, cuya primera fase está siendo puesta a punto en el Laboratorio Subterráneo de Canfranc (LSC) es detectar un tipo de desintegración extremadamente raro del Xe-136, llamado desintegración doble beta sin neutrinos. La observación de dichos procesos supondría la demostración de que el neutrino es su propia antipartícula, un descubrimiento con profundas consecuencias en física de partículas y cosmología, que tendría una relevancia comparable al descubrimiento de las oscilaciones de neutrinos o del bosón de Higgs (hallazgos que han sido galardonados con un premio Nobel). El desarrollo de NEXT fue financiado hasta 2014 por proyectos del MICYNT/MINECO así como por un proyecto CONSOLIDER-INGENIO 2010 y ha obtenido en 2014 un prestigioso Advanced Grant/ERC. 

~Por otra parte, durante el año 2015, el IP de este proyecto ha propuesto la aplicación de la tecnología desarrollada para NEXT a un nuevo tipo de detector PET para imagen médica, basado en el xenón líquido llamado PETALO. Este detector presenta dos grandes ventajas sobre los centelladores sólidos convencionales (tales como LSO). Por un lado es mucho más económico, lo que permitiría desarrollar aparatos PET de gran sensibilidad a bajo coste. Por otro, el tiempo de respuesta en centelleo es mucho más rápido, lo que permite una medida de tiempo de vuelo (TOF) excepcionalmente buena (con una resolución temporal de 100 ps o menos), lo que a su vez contribuye a una reducción espectacular de los ruidos de fondo característicos de la tecnología (coincidencias accidentales, dispersión Compton, etc.). La tecnología de PETALO, que ha sido protegida por una patente de la cual la UV es co-propietaria junto al CSIC, podría suponer una desarrollo extremadamente innovador en la tecnología PET.
 

{\bf ~El primer objetivo de este proyecto de investigación} es contribuir al desarrollo de NEXT, que podría ser el primer experimento de física de partículas realizado en España capaz de realizar o contribuir a un descubrimiento fundamental. 

{\bf ~El segundo objetivo de este proyecto de investigación} es contribuir de manera decisiva al desarrollo del proyecto PETALO, realizando una serie de medidas fundamentales para demostrar la viabilidad de la tecnología y su potencial para la construcción de aparatos clínicos. 