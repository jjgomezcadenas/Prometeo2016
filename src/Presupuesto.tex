La operación de los detectores NEW y NEXT-100 en el LSC requiere la presencia continuada de un equipo de científicos en Canfranc. En concreto, el IFIC es responsable de la coordinación de operación y técnica, mientras que la UPV es responsable de la coordinación de proyecto. Esto implica la presencia del coordinador de operaciones y el coordinador de proyecto durante 5 meses al año cada uno (normalmente ambos coordinadores alternan su presencia en el LSC). 

Los gastos de manutención (promediados durante periodos largos) se estiman en 2,000 euros por persona y mes. Por tanto, los gastos de operación para NEXT se estiman en 20,000 euros al año (dos personas durante 5 meses por año). 

El punto de sinergia más claro entre el proyecto NEXT y el proyecto PETALO es la electrónica. Los detectores P2 y PDEMO requieren ASIC especializados para medir el tiempo de vuelto. En concreto, se planea utilizar el 
TOFPET2 ASIC fabricado por PETSYS\footnote{\url{http://www.petsyselectronics.com/web/website/docs/products/product1/Flyer\%20ASIC\%20TOFPETv2.pdf}}. Se trata de un nuevo chip, con 64 canales, especialmente diseñado para aplicaciones TOF. Sus características le hacen ideal para su uso en P2 y PDEMO, {\bf pero además este ASIC podría sustituir a la electrónica COTS de NEXT}, de cara a una cuarta fase del proyecto. 

En consecuencia, se solicita financiación para un ingeniero electrónico durante los cuatro años del proyecto (coste estimado 40,000 euros al año) cuya actividad principal estará relacionada con la implementación de las soluciones ASIC tanto para PETALO como para NEXT. Además, este ingeniero jugará un rol decisivo en la puesta a punto de P2 (puesta a punto de SiPMs, adquisición de datos, etc.), así como en la mejora del plano de trazado de NEW utilizando la electrónica ASIC. 

El detector P2 se construirá en 2016. Su coste se estima en 30,000 \euro. P2 requiere la construcción de dos pequeños criostatos para albergar de manera independiente las dos LXSC que utilizará el sistema (coste estimado de los dos criostatos y la mecánica asociada 15,000 \euro). El coste del módulo PB (2 caras instrumentadas) se estima en 2,500 \euro. El coste del módulo PA (6 caras instrumentadas) se estima en 7,500 \euro. El coste total de PA y PB es 10,000 \euro. Se estiman en 5,000 \euro\ los gastos asociados a infraestructuras, fuente radioactiva, etc. 

En 2017 se construirá el criostato de PDEMO, diseñado para albergar un anillo de 14 módulos y cuyo coste se estima en 20,000 \euro. Se prevén además 10,000 \euro\ para construir los primeros 4 módulos de PDEMO.

En 2018 se construirán los diez módulos restantes de PDEMO (25,000 \euro). Se prevén 5,000 \euro\ para la adquisición de dispositivos de prueba para PDEMO (fantasmas NEMA). 

En 2019 se prevén 30,000 \euro\ para la mejora de 1,600 canales de electrónica en el plano de trazado de NEW. Se prevé un coste de 20,000 \euro\ para los ASIC y 10,000 \euro\ para la infraestructura asociada (fibra óptica, pasamuro para fibra óptica, módulos externos, etc.).

En resumen, el gasto puede desglosarse, de manera uniforme, durante los cuatro años del proyecto de la siguiente manera:
\begin{enumerate}
\item {\bf Personal}. Un ingeniero compartido entre NEXT y PETALO (40,000 \euro\ año).
\item {\bf Viajes y alojamiento}. Operación de NEXT en el LSC, 2 personas, a 5 meses por persona y 2,000 \euro\ al mes para manutención, alojamiento y viaje (20,000 \euro\ año).
\item {\bf Fungible}. Construcción de P2, PDEMO y mejora de la electrónica de NEXT (40,000 \euro).

\end{enumerate}
