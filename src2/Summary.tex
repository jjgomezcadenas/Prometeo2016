PETALO (PET with TOF Applications based in Liquid XenOn) is a novel technology for PET imaging based in liquid xenon (LXe) read by silicon photomultipliers (SiPMs) and  low power, low noise ASICs for Time of Flight applications. The basic element of PETALO is a liquid xenon scintillating cell, optimized to maximize the number of gammas that interact in the cell and to minimize pile-up. 

Xenon is a noble gas which scintillates as response to the ionizing radiation. The scintillation is very fast (2.2 ns in its fastest mode) and very intense (30,000 photons per 511 keV gamma). The combination of both features results in the possibility of building a PET of good energy and spatial resolution and excellent time resolution. This, in turn, makes it possible the application of PETALO as PET-TOF. The use of time of flight to reduce imaging errors is known since the beginning of the technology and the last few years have witnessed a renewed interest in the technique, including the introduction in the market of a commercial model of 600 ps CRT (Coincidence Resolving Time). Other recent devices feature improved CRT near 400 ps. PETALO could achieve CRTs in the range of 100 ps, thus representing a break-through in the technology.  The low cost of the LXe compared with conventional PET detectors such as LSO/LYSO opens the possibility to apply the technology to the construction of a system of large axial acceptance, suitable for ``full body'' PET applications. 

This project proposes crucial contributions to the development of PETALO, namely: i) the measurement of poorly know properties of LXe, crucial  for PET applications: ii) 
A characterization of the energy resolution and spatial resolution of the single scintillator cell; iii) a detailed measurement of the CRT resolution of the cell; and  iv) a  detailed calculation of the sensitivities expected for a ``full body'' PETALO scanner, (optimized for low cost) and a ``brain" PETALO-TOF scanner optimized to achieve the best possible CRT. 
