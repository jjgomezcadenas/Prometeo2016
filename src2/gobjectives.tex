\noindent\textbf{C.1.2 Starting hypothesis and general objectives}


\subsubsection*{Starting hypothesis: the PETALO concept}

Generically, PETALO is a PET scanner built by one of more rings of LXSCs, housed inside a cryostat. The technology can be optimized for different applications. For example, a PETALO scanner intended as a large acceptance ``full body'' PET, of $\sim$ 50 cm axial length and some 80 cm diameter could be built stacking 10 rings, each one made of about 50 large cells of $5 \times 5 \times 5$~cm$^3$~with (relatively) sparse SiPM instrumentation (32 SiPM per face, for a total of 64 SiPM per LXSC). The scanner would have lower cost than a LSO scanner (the LXe would cost 250 k\euro\ instead of 1.25 M\euro\, and the cost of the sensors and the electronics could also be halved) and better sensitivity (due to the much improved CRT).  On the other hand, a PETALO scanner intended for neuroimaging could have an axial length of some 20 cm (4 rings) and a radius of some 30 cm. This application can increase the number of channels (e.g., reduce the size of the SiPMs) to optimize the spatial resolution and the CRT.
Last but not least, the PETALO technology is expected to be fully compatible with MRI, thus making it suitable for combined MRI-PET scanners. 

The demonstration of the potential of PETALO requires the construction of a demonstrator deploying a full ring of LXSCs. This researcher is part of the {\em working team (equipo de trabajo)} of a proposal which has been submitted to the RETOS 2015 call along this lines\footnote{Development of a new PET apparatus based on liquid
xenon and compatible with NMR, PI. Prof. J. D\'iaz Medina.}. 

\subsubsection*{General objectives}

The general objective of this project is to realize crucial contributions to the development of the PETALO concept  {\bf necessary to demonstrate its feasibility and to assess its performance}. This contribution include, as specific goals: i) the measurement of poorly know properties of LXe, crucial  for PET applications: ii) 
a characterization of the energy resolution and spatial resolution of the LXSC; iii) a detailed measurement of the CRT resolution of the LXSC; iv) a  detailed calculation of the sensitivities expected for a ``full body'' PETALO scanner, (optimized for low cost) and a ``brain" PETALO-TOF scanner optimized to achieve the best possible CRT. For a more detailed description of the specific objectives, see next section.

 The project will make extensive use of the infrastructures and technological developments already achieved by the NEXT project, in which this researcher has been working for the last five years. NEXT is a high pressure xenon TPC for neutrinoless double beta decay searches, with separated readout for energy measurement and tracking. While the former is carried out with PMTs, the latter uses a matrix of SiPMs, coated with TPB for high light collection. This researcher has acquired extensive experience working with SiPMs. In particular, she has led the development of the reconstruction algorithms which, starting from the SiPM response in time, reconstruct the position of the electron tracks in the sensitive volume of the detector. This effort has recently produced a paper\footcite{Ferrario:2015kta}, accepted for publication in the Journal of High Energy Physics, with the PI as first author, which demonstrates the tracking capabilities of the NEXT experiment using SiPMs. 

%
%As far as concerns the hardware part, the cryostat needed to keep xenon at liquid phase will be fabricated by the same specialised company that has developed the vacuum/pressure chambers of NEXT, the gas system will reuse the components available from the NEXT prototypes and the boards that support the SiPMs and feedthroughs of the cells will be straight-forward modifications of those developed for the NEXT detector. The electronics equipment will also be available from the NEXT experiment system. The people in charge of this part of the project (one mechanical engineer and two electronic engineers) are currently employed at the Instituto de F\'isica Corpuscular in Valencia and will remain so throughout the whole duration of the project.
%
%This project will benefit from the collaboration of Prof. J.J. G\'omez Cadenas from CSIC, who holds an Advanced Grant and has applied for a Proof of Concept Grant in 2015 to build a full ring LXe PET and Prof. Jos\'e F. Toledo from the Universidad Polit\'ecnica de Valencia, who will lead the development of the electronics and the acquisition system of the cells. This collaboration will extend to international groups, such as the PET group at the Institute of Neuroscience and Medicine in Juelich (Germany), lead by Dr.  Christoph Lerche\footnote{\url{http://www.fz-juelich.de/inm/inm-4/EN/Forschung/PET-Physik/\_node.htm}} and the Biomedical Engineering group at the Institute for Biomedical Engineering of the Otto-von-Guericke University in Magdeburg, with Dr. Paola Solevi\footnote{\url{http://www.lms.ovgu.de}}. In Valencia, the GIBI230 group, working at the hospital La Fe (Valencia), led by the prestigious radiologist Dr. Luis Mart\'i-Bonmat\'i\footnote{\url{http://www.iislafe.es/biomedica-imagen.aspx}} will provide support for the tests of compatibility of this project with MRI using a research MRI device available at the hospital and will help the PI in the evaluation of clinical applications.
%
%

\paragraph{Match to the Spanish Strategy of Science and Technology and Innovation.}

This research project is presented within the program of ``Challenges of society'', specifically, challenge number 1: \emph{Health, demographic change and welfare'}. The objectives presented in this project are very well aligned to the Spanish program for science. They involve a direct technology transfer from a basic science experiment in particle physics to a medical imaging application of high social impact. It involves a multidisciplinary collaboration between physicists, engineers and medical doctors and it has a large potential for generating commercial spinoffs, including the possibility of a breakthrough in  PET-TOF, PET-NMR technology.  
\\
\\