\\
PETALO (PET con Aplicaciones de tiempo de vuelo (TOF, por sus siglas en ingl\'es) basado en xen\'on l\'iquido) es una nueva tecnolog\'ia para un escaneo PET basado en xen\'on l\'iquido (LXe) y le\'ido por fotomultiplicadores de silicio (SiPMs) y ASICs de baja potencia y ruido para aplicaciones de tiempo de vuelo. El elemento b\'asico de PETALO es una celda centelleadora de xen\'on l\'iquido, optimizada para maximizar el n\'umero de gammas que interact\'uan en la celda y minimizar la acumulaci\'on de eventos.

El xen\'on es un gas noble que centellea en respuesta a radiaci\'on ionizante. El centelleo es muy r\'apido (2.2 ns en su modo m\'as r\'apido) y muy intenso (30 000 fotones para un gamma de 511 keV). La combinaci\'on de ambas caracter\'isticas da lugar a la posibilidad de construir un PET con buena resoluci\'on energ\'etica y espacial y excelente resoluci\'on temporal. Esto, a su vez, hace posible la aplicaci\'on de PETALO como un PET-TOF. El uso del TOF para reducir los errores en las t\'ecnicas de reconstrucci\'on de imagen se conoce desde el comienzo de esta tecnolog\'ia y los \'ultimos a\~nos han visto renacer el inter\'es, con la introducci\'on en el mercado de un modelo comercial de 600 ps de tiempo de resoluci\'on de las coincidencias (CRT, por sus siglas en ingl\'es). Otros dispositivos recientes han mejorado en CRT hasta casi 400 ps. PETALO podr\'ia alcanzar CRTs en el rango de 100 ps, representando, por lo tanto, un gran avance en la tecnolog\'ia. El bajo coste del LXe comparado con los detectores PET convencionales como LSO/LYSO abre la posibilidad de aplicar la tecnolog\'ia a la construcci\'on de un sistema de amplia aceptancia axial, apropiada para aplicaciones PET a cuerpo entero.

Este proyecto propone contribuciones cruciales para el desarrollo de PETALO, esto es: i) la medida de las propiedades del LXe, que hoy en d\'ia se conocen con alta imprecisi\'on, cruciales para las aplicaciones PET; ii) una caracterizaci\'on de la resoluci\'on de energ\'ia y espacial de la celda centelleadora; iii) una medida detallada de la resoluci\'on del CRT de la celda; iv) un c\'alculo detallado de la sensitividad esperada para en esc\'aner PETALO a cuerpo entero (optimizado para bajo coste) y un esc\'aner de cerebro PETALO-TOF optimizado para alcanzar el mejor CRT posible.

