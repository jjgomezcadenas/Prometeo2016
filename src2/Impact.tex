\noindent\textbf{C.2.1 Scientific and technological impact}
%\subsubsection{Scientific and technological impact}

%\subsubsection{a. Economic and/or societal benefits}

The main application of PET to medicine is oncology, and, in recent years, neuroimaging. The oncological applications are mostly based in the use of FDG, a glucose analog where an atom of fluorine-18 replaces an atom of oxygen. 
A typical dose of FDG used in an oncological scan has an effective radiation dose of 14 mSv, equivalent to about 5 years of dose by background radiation (combining natural and artificial sources). It follows that one of the important issues in PET applications is to reduce the FDG dose as much as possible, which, in turn, requires an axial coverage as large as possible. Today's scanner have axial lengths of 15-25 cm. A larger scanner improves the solid angle, resulting in a larger acceptance and therefore allowing a dose reduction. Unfortunately {\bf the cost of current PET scanners is very high}, limiting their dimensions, and in particular its axial length. The fact that  individual PET scans are more expensive than ``conventional'' imaging with computed tomography (CT) and magnetic resonance imaging (MRI) limits also their use in clinical diagnose. {\bf Reducing costs of PET scanners is, therefore, a major priority to facilitate the expansion of the technology}. Conversely, the combination of PET with CT and MRI often leads to much improved scans, since the structural information offered by CT and MRI can be combined with the functional information offered by PET. A PETALO scanner intended as a large acceptance ``full body'' PET would have lower cost and better sensitivity (due to the much improved CRT) than a LYSO scanner.  

PET neuroimaging uses the fact that the brain is an avid user of glucose, and since brain pathologies such as Alzheimer's disease greatly decrease brain metabolism of glucose standard FDG-PET of the brain, which measures regional glucose use, may also be successfully used to differentiate Alzheimer's disease from other dementing processes, and also to make early diagnosis of Alzheimer's disease. Neuroimaging requires images as clear as possible, and therefore high--performance PET scanners fully compatible with MRI (thus capable of operating in very intense, highly variable magnetic fields). Using time-of-flight information is one of the best ways of boosting performance, insofar as the sensitivity of the scanner improves with $D/CTR$~, where $D$~is the diameter of the object being imaged (e.g, the head). An order of magnitude improvement in sensitivity can be reached by using a PETALO scanner with a CRT of 100 ps. 

Last but not least, the PETALO technology is expected to be fully compatible with MRI, thus making it suitable for combined MRI-PET scanners. 

Overall, the researcher expects to make major contributions to the development of the novel PETALO technology, orienting her career towards instrumentation applied to medical imaging and contributing to technology and human capital transfer from the basic research field in which she has invested her training years. 
\\

\noindent\textbf{C.2.2 Dissemination of results}
%\subsubsection{Dissemination of results}

The results of this project will be published as scientific papers in international journals with high impact factor, as well as presented in international conferences. The success of the project could bring to the launching of a startup technological company to fabricate and commercialize PETALO scanners and/or the acquisition of the technology by one of the larger PET manufacturers. 
\\